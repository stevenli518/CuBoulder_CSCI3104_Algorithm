 \documentclass[11pt]{article} 
\usepackage[english]{babel}
\usepackage[utf8]{inputenc}
\usepackage[margin=0.5in]{geometry}
\usepackage{amsmath}
\usepackage{amsthm}
\usepackage{amsfonts}
\usepackage{amssymb}
\usepackage[usenames,dvipsnames]{xcolor}
\usepackage{graphicx}
\usepackage[siunitx]{circuitikz}
\usepackage{tikz}
\usetikzlibrary{calc,arrows.meta}
\usepackage[colorinlistoftodos, color=orange!50]{todonotes}
\usepackage{hyperref}
\usepackage[numbers, square]{natbib}
\usepackage{fancybox}
\usepackage{epsfig}
\usepackage{soul}
\usepackage[framemethod=tikz]{mdframed}
\usepackage[shortlabels]{enumitem}
\usepackage[version=4]{mhchem}
\usepackage{multicol}
\usepackage{mathtools}
\usepackage{comment}
\usepackage{enumitem}
\usepackage[utf8]{inputenc}
\usepackage{listings}
\usepackage{color}
\usepackage[numbers]{natbib}
\usepackage{subfiles}
\usepackage{tkz-berge}
\usepackage{algorithm}
\usepackage[noend]{algpseudocode}


\newtheorem{prop}{Proposition}[section]
\newtheorem{thm}{Theorem}[section]
\newtheorem{lemma}{Lemma}[section]
\newtheorem{cor}{Corollary}[prop]

\theoremstyle{definition}
\newtheorem{definition}{Definition}

\theoremstyle{definition}
\newtheorem{required}{Problem}

\theoremstyle{definition}
\newtheorem{ex}{Example}

\tikzset{
	vertex/.style={circle,draw,minimum size=16, inner sep=0pt,font=\normalsize},
	every node/.style={draw=none,rectangle,font=\scriptsize,outer sep=0pt,inner sep=2pt},
	directed/.style={arrows={-Stealth[length=7pt]},font=\small},
	caption/.style={text width=6cm,align=center,rectangle,draw}
}


\setlength{\marginparwidth}{3.4cm}
%#########################################################

%To use symbols for footnotes
\renewcommand*{\thefootnote}{\fnsymbol{footnote}}
%To change footnotes back to numbers uncomment the following line
%\renewcommand*{\thefootnote}{\arabic{footnote}}

% Enable this command to adjust line spacing for inline math equations.
% \everymath{\displaystyle}

% _______ _____ _______ _      ______ 
%|__   __|_   _|__   __| |    |  ____|
%   | |    | |    | |  | |    | |__   
%   | |    | |    | |  | |    |  __|  
%   | |   _| |_   | |  | |____| |____ 
%   |_|  |_____|  |_|  |______|______|
%%%%%%%%%%%%%%%%%%%%%%%%%%%%%%%%%%%%%%%

\title{
\normalfont \normalsize 
\textsc{CSCI 3104 Spring 2022 \\ 
Instructors: Profs. Chen and Layer} \\
[10pt] 
\rule{\linewidth}{0.5pt} \\[6pt] 
\huge Quiz 20 - DP: Write down recurrence \\
\rule{\linewidth}{2pt}  \\[10pt]
}
%\author{Your Name}
\date{}

\begin{document}
\definecolor {processblue}{cmyk}{0.96,0,0,0}
\definecolor{processred}{rgb}{200, 0, 0}
\definecolor{processgreen}{rgb}{0, 255, 0}
\DeclareGraphicsExtensions{.png}
\DeclareGraphicsExtensions{.gif}
\DeclareGraphicsExtensions{.jpg}

\maketitle


%%%%%%%%%%%%%%%%%%%%%%%%%
%%%%%%%%%%%%%%%%%%%%%%%%%%
%%%%%%%%%%FILL IN YOUR NAME%%%%%%%
%%%%%%%%%%AND STUDENT ID%%%%%%%%
%%%%%%%%%%%%%%%%%%%%%%%%%%
\noindent
Due Date \dotfill April 1 \\
Name \dotfill \textbf{Chengming Li} \\
Student ID \dotfill \textbf{109251991} \\


\tableofcontents

\section{Instructions}
 \begin{itemize}
	\item The solutions \textbf{should be typed}, using proper mathematical notation. We cannot accept hand-written solutions. \href{http://ece.uprm.edu/~caceros/latex/introduction.pdf}{Here's a short intro to \LaTeX.}
	\item You should submit your work through the \textbf{class Canvas page} only. Please submit one PDF file, compiled using this \LaTeX \ template.
	\item You may not need a full page for your solutions; pagebreaks are there to help Gradescope automatically find where each problem is. Even if you do not attempt every problem, please submit this document with no fewer pages than the blank template (or Gradescope has issues with it).

	\item You \textbf{may not collaborate with other students}. \textbf{Copying from any source is an Honor Code violation. Furthermore, all submissions must be in your own words and reflect your understanding of the material.} If there is any confusion about this policy, it is your responsibility to clarify before the due date. 

	\item Posting to \textbf{any} service including, but not limited to Chegg, Discord, Reddit, StackExchange, etc., for help on an assignment is a violation of the Honor Code.
\end{itemize}

\newpage
\section{Standard 20 - DP: Write down Recurrence}

\begin{required}
Suppose we have an $m$-letter alphabet, $\Sigma = \{0, 1, \ldots, m-1\}$. Determine a recurrence for the number of strings $\omega$ of length $n$, such that no two consecutive characters in $\omega$ are the same. Clearly justify your recurrence.
\end{required}

\begin{proof}[Answer]
%Your answer here
\begin{itemize}
\item if n = 0, it is an empty string, so 0 string.
\item if n = 1, there is m combinations because it is impossible to have two characters in strings $\omega$
\item $f_2 = |W_2|  = |\Sigma^{2}| = \{0, 1, \ldots, m-1\} \cdot \{1, \ldots, m-1\}= m\cdot (m-1) + (m-1).$ The "m-1" term is due to $\omega$ does not have $00$ as a substring. For the first string,\{0, 1, \ldots, m-1\}, we have m combinations. And for each combination in the first string, the second string, $\{1, \ldots, m-1\}$, can give(m-1) combinations with that. That is why we have $m\cdot (m-1)$ term as the first term. The first term only eliminates the possibility of the first $0$ occurrence. So we need to add another (m-1) term to eliminate the possibility of the second $0$ occurrence.
\item $f_3 = |W_3|  = |\Sigma^{3}| = \{0, 1, \ldots, m-1\} \cdot \{1, \ldots, m-1\} \cdot \{1, \ldots, m-1\}= (m-1)f_2 + (m-1)f_1.$ The "m-1" term is due to $\omega$ does not have $00$ as a substring.
\item Same thing apply to the general case. (m-1) is used to eliminate the possibility of $0$ occurrence. And we multiply the result to $f_{n-1}$ combinations, which we already known the solution of W that doesn't have substring $00$. And same rule apply to the result to $f_{n-2}$ combinations.

\end{itemize}
\begin{align*}
f_{n} &= \begin{cases} \text{0} & : \text{n = 0}, \\ 
\text{m} & : \text{n = 1},\\
\text{$f_1\cdot (m-1) + (m-1)$} & : \text{n = 2}.,\\
\text{$f_{n-1}\cdot (m-1) + f_{n-2}\cdot(m-1)$} & : \text{$n \geq 3$}.,\\
\end{cases}
\end{align*}
\end{proof}


%LaTeX code to render a recurrence
\begin{comment}
\begin{align*}
f_{n} &= \begin{cases}
\text{case1} & : \text{condition1}, \\
\text{case2} & : \text{condition2}. 
\end{cases}
\end{align*}
\end{comment}

%%%%%%%%%%%%%%%%%%%%%%%%%%%%%%%%%%%%%%%%%%%%%%%%%%
\end{document} % NOTHING AFTER THIS LINE IS PART OF THE DOCUMENT



