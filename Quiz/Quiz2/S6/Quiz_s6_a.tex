\documentclass[11pt]{article} 
\usepackage[english]{babel}
\usepackage[utf8]{inputenc}
\usepackage[margin=0.5in]{geometry}
\usepackage{amsmath}
\usepackage{amsthm}
\usepackage{amsfonts}
\usepackage{amssymb}
\usepackage[usenames,dvipsnames]{xcolor}
\usepackage{graphicx}
\usepackage[siunitx]{circuitikz}
\usepackage{tikz}
\usepackage[colorinlistoftodos, color=orange!50]{todonotes}
\usepackage{hyperref}
\usepackage[numbers, square]{natbib}
\usepackage{fancybox}
\usepackage{epsfig}
\usepackage{soul}
\usepackage[framemethod=tikz]{mdframed}
\usepackage[shortlabels]{enumitem}
\usepackage[version=4]{mhchem}
\usepackage{multicol}

\usepackage{mathtools}
\usepackage{comment}
\usepackage{enumitem}
\usepackage[utf8]{inputenc}
\usepackage[linesnumbered,ruled,vlined]{algorithm2e}
\usepackage{listings}
\usepackage{color}
\usepackage[numbers]{natbib}
\usepackage{subfiles}
\usepackage{tkz-berge}


\newtheorem{prop}{Proposition}[section]
\newtheorem{thm}{Theorem}[section]
\newtheorem{lemma}{Lemma}[section]
\newtheorem{cor}{Corollary}[prop]

\theoremstyle{definition}
\newtheorem{definition}{Definition}

\theoremstyle{definition}
\newtheorem{required}{Problem}

\theoremstyle{definition}
\newtheorem{ex}{Example}


\setlength{\marginparwidth}{3.4cm}
%#########################################################

%To use symbols for footnotes
\renewcommand*{\thefootnote}{\fnsymbol{footnote}}
%To change footnotes back to numbers uncomment the following line
%\renewcommand*{\thefootnote}{\arabic{footnote}}

% Enable this command to adjust line spacing for inline math equations.
% \everymath{\displaystyle}

% _______ _____ _______ _      ______ 
%|__   __|_   _|__   __| |    |  ____|
%   | |    | |    | |  | |    | |__   
%   | |    | |    | |  | |    |  __|  
%   | |   _| |_   | |  | |____| |____ 
%   |_|  |_____|  |_|  |______|______|
%%%%%%%%%%%%%%%%%%%%%%%%%%%%%%%%%%%%%%%

\title{
\normalfont \normalsize 
\textsc{CSCI 3104 Spring 2022 \\ 
Instructor: Profs. Chen and Layer} \\
[10pt] 
\rule{\linewidth}{0.5pt} \\[6pt] 
\huge Quiz 6 - Dijkstra's \\
\rule{\linewidth}{2pt}  \\[10pt]
}
%\author{Your Name}
\date{}

\begin{document}

\maketitle


%%%%%%%%%%%%%%%%%%%%%%%%%
%%%%%%%%%%%%%%%%%%%%%%%%%%
%%%%%%%%%%FILL IN YOUR NAME%%%%%%%
%%%%%%%%%%AND STUDENT ID%%%%%%%%
%%%%%%%%%%%%%%%%%%%%%%%%%%
\noindent
Due Date \dotfill February 11 \\
Name \dotfill \textbf{Chengming Li} \\
Student ID \dotfill \textbf{109251991} \\


\tableofcontents

\section{Instructions}
 \begin{itemize}
	\item The solutions \textbf{should be typed}, using proper mathematical notation. We cannot accept hand-written solutions. \href{http://ece.uprm.edu/~caceros/latex/introduction.pdf}{Here's a short intro to \LaTeX.}
	\item You should submit your work through the \textbf{class Canvas page} only. Please submit one PDF file, compiled using this \LaTeX \ template.
	\item You may not need a full page for your solutions; pagebreaks are there to help Gradescope automatically find where each problem is. Even if you do not attempt every problem, please submit this document with no fewer pages than the blank template (or Gradescope has issues with it).

	\item You \textbf{may not collaborate with other students}. \textbf{Copying from any source is an Honor Code violation. Furthermore, all submissions must be in your own words and reflect your understanding of the material.} If there is any confusion about this policy, it is your responsibility to clarify before the due date. 

	\item Posting to \textbf{any} service including, but not limited to Chegg, Discord, Reddit, StackExchange, etc., for help on an assignment is a violation of the Honor Code.

	\item You \textbf{must} virtually sign the Honor Code (see Section \ref{HonorCode}). Failure to do so will result in your assignment not being graded.
\end{itemize}


\section{Honor Code (Make Sure to Virtually Sign)} \label{HonorCode}

\begin{required}
\noindent 
\begin{itemize}
\item My submission is in my own words and reflects my understanding of the material.
\item I have not collaborated with any other person.
\item I have not posted to external services including, but not limited to Chegg, Discord, Reddit, StackExchange, etc.
\item I have neither copied nor provided others solutions they can copy.
\end{itemize}

%\noindent In the specified region below, clearly indicate that you have upheld the Honor Code. Then type your name. 
\end{required}

\begin{proof}[Agreed (signature here).]
%% Typing "I agree to the above," followed by your name is sufficient.
I agree to the above, Chengming Li
\end{proof}



\newpage
\section{Standard 6 - Dijkstra's}

\subsection{Problem \ref{Dijkstra1}}
\begin{required} \label{Dijkstra1}
Consider the weighted graph $G(V, E, w)$ pictured below. Work through Dijkstra's algorithm on the following graph, using the source vertex $A$. 
\begin{itemize}
\item Clearly include the contents of the priority queue, as well as the distance from $A$ to each vertex at each iteration.
\item If you use a table to store the distances, clearly label the keys according to the vertex names rather than numeric indices (i.e., \texttt{dist[`B']} is more descriptive than \texttt{dist[`1']}).
\item You do \textbf{not} need to draw the graph at each iteration, though you are welcome to do so. [This may be helpful scratch work, which you do not need to include.]
\end{itemize}

\definecolor {processblue}{cmyk}{0.96,0,0,0}
\begin{center}
	\begin {tikzpicture}[-latex ,auto ,node distance =2 cm and 3cm ,on grid ,
	semithick ,
	state/.style ={ circle ,top color =white , bottom color = processblue!20 ,
	draw,processblue , text=blue , minimum width =1 cm}, scale=0.5]

	\node[state] (A) {$A$};
	\node[state] (B) [above right = of A] {$B$};
	\node[state] (C) [below right = of A] {$C$};
	\node[state] (G) [below right = of B] {$F$};
	\node[state] (D) [above right = of G] {$D$};
	\node[state] (E) [below right = of G] {$E$};
	
	
	\path (A) edge node[above] {$6$} (B);
	\path (A) edge node[above] {$11$} (C);
	\path (B) edge node[above] {$1$} (G);
	\path (B) edge node[above] {$6$} (D);
	\path (G) edge node[above] {$1$} (A);
	\path (E) edge node[above] {$1$} (C);
	\path (G) edge node[above] {$3$} (D);	
	\path (G) edge node[above] {$5$} (E);
	\path (G) edge node[above] {$2$} (C);
	
	\end{tikzpicture}  
\end{center}

\end{required}

\begin{proof}[Answer]
%Your answer here
\begin{center}
\begin{tabular}[c]{|c|c|c|} 
	Vertex &Dist from A &Prev-Vertex \\\hline
	A &0 & NULL\\
	B &$\infty$ & NULL\\
	C &$\infty$ & NULL\\
	D &$\infty$ & NULL\\
	E &$\infty$ & NULL\\
	F &$\infty$ & NULL\\
\end{tabular}
\end{center}	
\textbf{PQ = [(A,0)]}

\begin{center}
\begin{tabular}[c]{|c|c|c|} 
	Vertex &Dist from A &Prev-Vertex \\\hline
	A &0 & NULL\\
	B &6 & A\\
	C &11& A\\
	D &$\infty$ & NULL\\
	E &$\infty$ & NULL\\
	F &$\infty$ & NULL\\
\end{tabular}
\end{center}	
\textbf{PQ = [(B,6),(C,11)]}

\begin{center}
\begin{tabular}[c]{|c|c|c|} 
	Vertex &Dist from A &Prev-Vertex \\\hline
	A &0 & NULL\\
	B &6 & A\\
	C &11& A\\
	D &12 & B\\
	E &$\infty$ & NULL\\
	F &7 & B\\
\end{tabular}
\end{center}	

\textbf{PQ = [(F,7),(C,11),(D,12)]}

\begin{center}
\begin{tabular}[c]{|c|c|c|} 
	Vertex &Dist from A &Prev-Vertex \\\hline
	A &0 & NULL\\
	B &6 & A\\
	C &9& F\\
	D &10 & F\\
	E &12 & F\\
	F &7 & B\\
\end{tabular}
\end{center}	

\textbf{PQ = [(C,9),(D,10),(E,12)]}

\begin{center}
\begin{tabular}[c]{|c|c|c|} 
	Vertex &Dist from A &Prev-Vertex \\\hline
	A &0 & NULL\\
	B &6 & A\\
	C &9& F\\
	D &10 & F\\
	E &12 & F\\
	F &7 & B\\
\end{tabular}
\end{center}	

\textbf{PQ = [(D,10),(E,12)]}

\begin{center}
\begin{tabular}[c]{|c|c|c|} 
	Vertex &Dist from A &Prev-Vertex \\\hline
	A &0 & NULL\\
	B &6 & A\\
	C &9& F\\
	D &10 & F\\
	E &12 & F\\
	F &7 & B\\
\end{tabular}
\end{center}	

\textbf{PQ = [(E,12)]}

\begin{center}
\begin{tabular}[c]{|c|c|c|} 
	Vertex &Dist from A &Prev-Vertex \\\hline
	A &0 & NULL\\
	B &6 & A\\
	C &9& F\\
	D &10 & F\\
	E &12 & F\\
	F &7 & B\\
\end{tabular}
\end{center}	

\textbf{PQ = [] Algorithm terminates}
\end{proof}







%%%%%%%%%%%%%%%%%%%%%%%%%%%%%%%%%%%%%%%%%%%%%%%%%%
\end{document} % NOTHING AFTER THIS LINE IS PART OF THE DOCUMENT



