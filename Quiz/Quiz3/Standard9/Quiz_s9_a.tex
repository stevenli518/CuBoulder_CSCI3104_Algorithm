\documentclass[11pt]{article} 
\usepackage[english]{babel}
\usepackage[utf8]{inputenc}
\usepackage[margin=0.5in]{geometry}
\usepackage{amsmath}
\usepackage{amsthm}
\usepackage{amsfonts}
\usepackage{amssymb}
\usepackage[usenames,dvipsnames]{xcolor}
\usepackage{graphicx}
\usepackage[siunitx]{circuitikz}
\usepackage{tikz}
\usepackage[colorinlistoftodos, color=orange!50]{todonotes}
\usepackage{hyperref}
\usepackage[numbers, square]{natbib}
\usepackage{fancybox}
\usepackage{epsfig}
\usepackage{soul}
\usepackage[framemethod=tikz]{mdframed}
\usepackage[shortlabels]{enumitem}
\usepackage[version=4]{mhchem}
\usepackage{multicol}

\usepackage{mathtools}
\usepackage{comment}
\usepackage{enumitem}
\usepackage[utf8]{inputenc}
\usepackage[linesnumbered,ruled,vlined]{algorithm2e}
\usepackage{listings}
\usepackage{color}
\usepackage[numbers]{natbib}
\usepackage{subfiles}
% \usepackage{tkz-berge}


\newtheorem{prop}{Proposition}[section]
\newtheorem{thm}{Theorem}[section]
\newtheorem{lemma}{Lemma}[section]
\newtheorem{cor}{Corollary}[prop]

\theoremstyle{definition}
\newtheorem{definition}{Definition}

\theoremstyle{definition}
\newtheorem{required}{Problem}

\theoremstyle{definition}
\newtheorem{ex}{Example}


\setlength{\marginparwidth}{3.4cm}
%#########################################################

%To use symbols for footnotes
\renewcommand*{\thefootnote}{\fnsymbol{footnote}}
%To change footnotes back to numbers uncomment the following line
%\renewcommand*{\thefootnote}{\arabic{footnote}}

% Enable this command to adjust line spacing for inline math equations.
% \everymath{\displaystyle}

% _______ _____ _______ _      ______ 
%|__   __|_   _|__   __| |    |  ____|
%   | |    | |    | |  | |    | |__   
%   | |    | |    | |  | |    |  __|  
%   | |   _| |_   | |  | |____| |____ 
%   |_|  |_____|  |_|  |______|______|
%%%%%%%%%%%%%%%%%%%%%%%%%%%%%%%%%%%%%%%

\title{
\normalfont \normalsize 
\textsc{CSCI 3104 Spring 2022\\
Instructor: Profs. Chen and Layer} \\
[10pt] 
\rule{\linewidth}{0.5pt} \\[6pt] 
\huge Quiz 9 - Prim's Algorithm \\
\rule{\linewidth}{2pt}  \\[10pt]
}
%\author{Your Name}
\date{}

\begin{document}
\definecolor {processblue}{cmyk}{0.96,0,0,0}
\definecolor{processred}{rgb}{200, 0, 0}
\definecolor{processgreen}{rgb}{0, 255, 0}
\maketitle


%%%%%%%%%%%%%%%%%%%%%%%%%
%%%%%%%%%%%%%%%%%%%%%%%%%%
%%%%%%%%%%FILL IN YOUR NAME%%%%%%%
%%%%%%%%%%AND STUDENT ID%%%%%%%%
%%%%%%%%%%%%%%%%%%%%%%%%%%
\noindent
Due Date \dotfill February 18 \\
Name \dotfill \textbf{Chengming Li} \\
Student ID \dotfill \textbf{109251991} \\


\tableofcontents

\section{Instructions}
 \begin{itemize}
	\item The solutions \textbf{should be typed}, using proper mathematical notation. We cannot accept hand-written solutions. \href{http://ece.uprm.edu/~caceros/latex/introduction.pdf}{Here's a short intro to \LaTeX.}
	\item You should submit your work through the \textbf{class Canvas page} only. Please submit one PDF file, compiled using this \LaTeX \ template.
	\item You may not need a full page for your solutions; pagebreaks are there to help Gradescope automatically find where each problem is. Even if you do not attempt every problem, please submit this document with no fewer pages than the blank template (or Gradescope has issues with it).

	\item You \textbf{may not collaborate with other students}. \textbf{Copying from any source is an Honor Code violation. Furthermore, all submissions must be in your own words and reflect your understanding of the material.} If there is any confusion about this policy, it is your responsibility to clarify before the due date. 

	\item Posting to \textbf{any} service including, but not limited to Chegg, Discord, Reddit, StackExchange, etc., for help on an assignment is a violation of the Honor Code.
\end{itemize}


\newpage
\section{Standard 9 - Prim's Algorithm}

\subsection{Problem \ref{prob}}
\begin{required} \label{prob}
Consider the following graph $G(V, E, w)$. Clearly indicate the order in which Prim's algorithm adds the edges to the minimum-weight spanning tree \textbf{using} $F$ \textbf{as the source vertex}. You may simply list the order of the edges; it is not necessary to exhibit the state of the algorithm at each iteration.

\begin{center}
\begin {tikzpicture}[semithick]
\tikzstyle{blue}=[circle ,top color =white , bottom color = processblue!20 ,draw,processblue , text=blue , minimum width =1 cm];
\tikzstyle{red}=[circle ,top color =white , bottom color = processred!20 ,draw, processred , text=blue , minimum width =1 cm];
\tikzstyle{green}=[circle ,top color =white , bottom color = processgreen!20 ,draw, processgreen , text=blue , minimum width =1 cm];

	\node[blue] (A) {$A$};
	\node[blue] (B) [above right = of A] {$B$};
	\node[blue] (C) [below right = of A] {$C$};
	\node[blue] (D) [right = of B] {$D$};
	\node[blue] (E) [right = of C] {$E$};
	\node[blue] (F) [below right = of D] {$F$};
	
	\path (A) edge node[above] {$2$} (B);
	\draw (A) edge node[right] {$8$} (C);
	\path (D) edge node[left] {$4$} (C);
	\path (B) edge node[above] {$7$} (D);
	\path (C) edge node[above] {$5$} (E);
	\draw (E) edge node[right] {$5$} (D);
	\path (D) edge node[above] {$7$} (F);
	\draw (E) edge node[below] {$2$} (F);
	\end{tikzpicture}  
\end{center}
\end{required}

% Either type your answer in below, or uncomment the \includegraphics command
% and use it to insert an approprate image. Try experimenting with the scale 
% 0.9 the width option to resize your image if necessary.

%\includegraphics[width=0.9\textwidth]{solution.jpg}

\begin{proof}[Answer]
The algorithm first initialize an intermediate spanning tree $\mathcal{F}$  with all the vertex in weighted graph G, but no edges in it. \\
Then, algorithm initialize an priority queue Q (ascending order) to contains the edge incident to the source vertex F.\\
So: Q:[(\{E,F \},2), (\{D,F \},7)]
\begin{itemize}
\item EF: Of the edges connected to F whose addition would not create a cycle, EF has the minimum weight and is added to the tree.\\
So: Q:[(\{C,E \},5), (\{D,E \},5),(\{D,F \},7)]
\item CE: Of the edges connected to F or E whose addition would not create a cycle, CE has the minimum weight and is added to the tree.\\
So: Q:[(\{C,D \},4), (\{D,E \},5),(\{D,F \},7),(\{A,C \},8)]
\item CD: Of the edges connected to F or E or C whose addition would not create a cycle, CD has the minimum weight and is added to the tree.\\
So: Q:[(\{D,E \},5),(\{D,F \},7),(\{B,D \},7),(\{A,C \},8)]
\item DE: Of the edges connected to F or E or C or D whose addition would create a cycle, So DE is not added to the tree.\\
So: Q:[(\{D,F \},7),(\{B,D \},7),(\{A,C \},8)]
\item DF: Of the edges connected to F or E or C or D whose addition would create a cycle, So DF is not added to the tree.\\
So: Q:[(\{B,D \},7),(\{A,C \},8)]
\item BD: Of the edges connected to F or E or C or D whose addition would not create a cycle, BD has the minimum weight and is added to the tree.\\
So: Q:[(\{A,B \},2),(\{A,C \},8)]
\item AB: Of the edges connected to F or E or C or D or B whose addition would not create a cycle,  AB has the minimum weight and is added to the tree.\\
So: Q:[(\{A,C \},8)]
\item AC: Of the edges connected to F or E or C or D or B or A whose addition would create a cycle, So AC is not added to the tree.\\
So: Q:[]
\item Algorithm terminates, because we have $6$ vertex and $6-1 = 5$ edges in  $\mathcal{F}$. Then algorithm will return $\mathcal{F}$ as result.
\item So Prim's algorithm added the edges in order: \{E, F\}, \{C, E\}, \{C, D\}, \{B, D\}, \{A, B\}.
\end{itemize}
\end{proof}


%Include an Image: \includegraphics{ImageFileName}
%Include an Image and Rotate 90 degree: \includegraphics[angle=90]{ImageFileName}
%Include an Image, Rotate by 180 degrees, and scale by 50\% \includegraphics[scale=0.5, angle=90]{ImageFileName}


%%%%%%%%%%%%%%%%%%%%%%%%%%%%%%%%%%%%%%%%%%%%%%%%%%
\end{document} % NOTHING AFTER THIS LINE IS PART OF THE DOCUMENT



